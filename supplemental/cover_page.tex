\normalsize%
\title{Evocube {-} Supplemental material}%
\date{}%
\maketitle%
\large%
This file includes all the supplemental material submitted along with \textit{Evocube: A Genetic Labelling Framework for Polycube-Maps}.%

%
\vspace{20pt}%
\textbf{Content. }%
For each mesh we visualize:%
\begin{itemize}%
\item%
Input shape%
\item%
Computed labelling%
\item%
Associated polycube base{-}complex%
\end{itemize}%

%
\vspace{20pt}%
\textbf{Data. }%
We use the following datasets:%
\begin{itemize}%
\item%
\href{https://gitlab.com/franck.ledoux/mambo}{MAMBO} (Challenging CAD models)%
\item%
\href{https://deep-geometry.github.io/abc-dataset/}{The ABC dataset} (General CAD models)%
\item%
Meshes used in \href{https://gaoxifeng.github.io/}{\textit{Feature Preserving Octree-Based Hexahedral Meshing}}. (Natural shapes)%
\end{itemize}%
%
\vspace{20pt}%

Given the size of ABC, we use a subset of 1000 models and we do not optimize our polycubes (we simply use the fast polycube estimator described in our paper). 
In our paper, we report aggregate statistics on each dataset, as well as comparison with previous works.

\normalsize%
\clearpage%
%
%